\documentclass[fleqn,10pt]{wlscirep}
\usepackage[utf8]{inputenc}
\usepackage[T1]{fontenc}
\title{Summary of OPERA simulations for the ambient magnetic compensation}

\author[1]{Takashi Higuchi}
\affil[1]{\href{mailto:thiguchi@rcnp.osaka-u.ac.jp}{thiguchi@rcnp.osaka-u.ac.jp}}

%\keywords{Keyword1, Keyword2, Keyword3}

\begin{abstract}
Summary of OPERA simulations of MSR saturation by cyclotron stray magnetic fields and design of compensation coils.
\end{abstract}
\begin{document}

\flushbottom
\maketitle


\thispagestyle{empty}


\section{About}

The Introduction section, of referenced text\cite{Figueredo:2009dg} expands on the background of the work (some overlap with the Abstract is acceptable). The introduction should not include subheadings.

\section*{Results}

Up to three levels of \textbf{subheading} are permitted. Subheadings should not be numbered.

\subsection*{Subsection}

Example text under a subsection. Bulleted lists may be used where appropriate, e.g.

\begin{itemize}
\item First item
\item Second item
\end{itemize}

\subsubsection*{Third-level section}
 
Topical subheadings are allowed.

\section*{Discussion}

The Discussion should be succinct and must not contain subheadings.

\section*{Methods}

Topical subheadings are allowed. Authors must ensure that their Methods section includes adequate experimental and characterization data necessary for others in the field to reproduce their work.

\bibliography{sample}


% \begin{figure}[ht]
% \centering
% \includegraphics[width=\linewidth]{stream}
% \caption{Legend (350 words max). Example legend text.}
% \label{fig:stream}
% \end{figure}

\begin{table}[ht]
\centering
\begin{tabular}{|l|l|l|}
\hline
Condition & n & p \\
\hline
A & 5 & 0.1 \\
\hline
B & 10 & 0.01 \\
\hline
\end{tabular}
\caption{\label{tab:example}Legend (350 words max). Example legend text.}
\end{table}

Figures and tables can be referenced in LaTeX using the ref command, e.g. Figure \ref{fig:stream} and Table \ref{tab:example}.


\end{document}